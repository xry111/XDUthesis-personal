\chapter{总结及进一步改进方向}

\section{论文的主要工作}

本论文介绍了太阳能电池的工作原理和结构,回顾了压缩感知理论的发展和应用
实例,介绍了基于压缩感知的太阳能电池缺陷检测技术的进展,并分析了其相对于
传统检测方式的优点。论文的主要工作如下:

(1) 论述了压缩感知方法检测太阳能电池缺陷的可行性和优点。

(2) 设计测试数据,对太阳能电池的常见缺陷形态进行建模。

(3) 完成了基于压缩感知的太阳能电池缺陷检测系统的理论设计,使用 \verb|TVAL3|
和 \verb|YALL1| 软件包对全变分最小化恢复和 $\ell_1$ 最小化恢复进行了仿真
验证。

(4) 分析了现有的压缩感知太阳能电池缺陷检测系统原型。

(5) 针对现有系统受到太阳能电池栅电极影响,稀疏度难以进一步提高的问题,通过
改造测量矩阵,成功减少了 $20\%$ 至 $40\%$ 的采样。

\section{下一步工作展望}

压缩感知方法检测太阳能电池缺陷是一个很新的课题,其中有众多内容值得研究。

\subsection{数学方向}

(1) 研究不同种类的噪声,特别是作用于测量矩阵的噪声对压缩感知稀疏恢复的
影响。

(2) 研究非抽取小波变换、脊波、曲波等冗余变换,进一步揭示太阳能电池缺陷
的稀疏性。

(3) 研究压缩感知优化模型及其求解算法中各个参数的意义及调节方法,尝试使其
自动调节。

\subsection{工程方向}

(1) 用编译型语言或硬件实现高性能的压缩感知稀疏恢复例程。

(2) 进一步降低压缩感知检测缺陷所需的系统资源,实现对太阳能电池板的高效、
快速缺陷检测。

由于本人的水平和时间有限,这些工作留待研究生学习中解决。
