\section{压缩感知理论简介}

传统的图像采集遵循 Nyquist-Shannon 采样定理:
\begin{theorem}[采样定理] \label{th:Nyquist}
若函数 $x(t)$ 不含有频率高于 $B$ 的分量,则它可由在一列间距为 $1/(2B)$
的时间点上的采样值完全确定。
\end{theorem}

尽管定理 \ref{th:Nyquist} 的描述是针对时域采样,但对于空间域上的图像采集,
这一定理仍然适用。在日常生活中,如果用 CCD 相机等采样系统拍摄 LCD 显示器
等光栅系统,若采样系统的采样率过低,就会注意到 Moir\'e 条纹,这就
是欠采样导致的混叠。因此,现实中,人们往往尽量提高图像采集的采样率,以避
免发生混叠,保留图像中的细节信息。

然而,满足定理 \ref{th:Nyquist} 的采样往往导致过多的冗余测量值,占据巨大
的存储空间。另一方面,在某些特定应用中,满足定理 \ref{th:Nyquist} 的采样
会耗费过多的成本和时间。为此,人们开始探索新的采样理论,近年来形成的
压缩感知 (Compressive Sensing) 或称压缩采样
(Compressive Sampling) 理论就是一种重要的新型采样理论。

本文的太阳能电池缺陷检测方法以压缩感知作为理论基础,下面对其进行介绍。

\subsection{稀疏性和可压缩性}

CS 理论指出,可以从 $M (M << N)$ 个测量值中恢复包含 $N$ 个数据样本的信号
和图像。从信息论的视角来看,只有信号中包含的信息量比其带宽小得多时,才可能
达到这一目标。这一事实可由信号的稀疏性或可压缩性反映。

可用 $\ell_0$ 伪范数衡量离散信号(向量)的稀疏性:

\begin{definition}[$\ell_0$ 伪范数] \label{def:l0} 向量 $x$ 的
$\ell_0$ 伪范数定义为 $\mathbf{x}$ 中非零元素的个数:
\begin{equation}
\|x\|_0 = |\{k:x_k \neq 0\}|
\end{equation}
\end{definition}
注意定义 \ref{def:l0} 并不满足齐次性,因此不能称为范数或拟范数。

给定了稀疏性的衡量标准后,即可定义哪些向量是稀疏的。
\begin{definition}[$K$-稀疏] \label{def:K-sparse} 若
$\|x\|_0 \leq K$ ,则称 $x$ 是 $K$-稀疏 的。
称所有 $K$-稀疏信号组成的集合为 $\Sigma_K$。
\end{definition}

由于噪声的影响,实际信号不可能是严格稀疏的。但是,如果噪声不是特别强,
许多实际信号都可以通过稀疏信号近似表达。一般地,在信号 $x$ 中,选取
$K$ 个幅值最大的元素,将其余元素置为 $0$ ,即可得到该信号在 $\Sigma_K$
中最优的近似表达 $\hat x$\cite{KeepK}。 近似表达的误差可以用 $\ell_p$
范数定量衡量。
\begin{definition}[$\ell_p$ 范数, $\ell_p$ 拟范数] \label{def:lp}
\begin{equation}
\|x\|_p = \left( \sum x_k^p \right)^{\frac{1}{p}}
\end{equation}
若 $p \geq 1$ ,则 $\|x\|_p$ 满足三角不等式,称为 $\ell_p$ 范数。
若 $0 < p < 1$, $\|x\|_p$ 不满足三角不等式,称为 $\ell_p$ 拟范数。
\end{definition}

近似稀疏表达的误差一般用 $\ell_2$ 范数 $\|x - \hat x\|_2$ 加以衡量。信号
的近似稀疏表达早在压缩感知理论之前就被人们所熟知,并广泛用于
图像处理\cite{SparseImage}、信号压缩、统计学\cite{lasso}等领域。

尽管太阳能电池的缺陷图像并非传统意义上的光学图像,但是在工程实践中,人们仍
使用传统的图像压缩算法对太阳能电池缺陷图像进行压缩存储,且取得了较好的压缩
率。这表明,太阳能电池缺陷图像和光学图像一样,都存在较好的近似稀疏表达。

若给定数字图像的分辨率 $H \times W$ ,则该图像可以表示为长度为 $H \times W$
的向量 $x$ ,该向量由 $H$ 个长度为 $W$ 的向量 $r_1, r_1, \cdots, r_H$
连接而成,$r_k$ 表示图像的第 $k$ 行的 $W$ 个像素点。尽管在某些应用中将图像
表示为矩阵,但在 CS 理论中使用一维信号 $x$ 表示图像更为合适(这也是计
算机内存中二维数组的存储方式),因为 CS 理论本身并不关心信号的拓扑结构。

一般来说,表示图像的信号 $x$ 本身十分稠密。但是在合适的变换域中,该信号可能
被稀疏化。例如,若图像是水平、垂直方向的正弦波之积,那么 $x$ 不是稀疏的,但
其二维 Fourier 变换是 $1$-稀疏的。更一般地,可以将 $x$ 表示为 $T$ 个基本
波形,或称为原子的线性组合,即
\begin{equation}
x = \Phi a = \sum_{i=1}^T a_i \phi_i
\end{equation}
其中 $a_i$ 为 $x$ 在字典 $\Phi = [\phi_1, \phi_2, \cdots, \phi_T]$
下的稀疏表示系数;$\Phi$ 为 $N \times T$ 的矩阵,其列向量为原子 $\phi_k$。
若 $a = [a_1, a_2, \cdots a_T]$ 可用 $K$-稀疏 ($K << N$) 的系数向量
$\hat a$ 近似表示,且每个原子的 $\ell_2$ 范数 $\|\phi_k\|_2$ 相同(一般应
对字典作归一化),则可以只存储 $\hat a$ 中的 $K$ 个非零元素,并以线性组合
\begin{equation}
\hat x = \Phi \hat a
\end{equation}
较好地近似表示 $x$ 。

字典中原子的个数 $T$ 可能大于,等于,或者小于信号的长度 $N$ 。对于 Fourier
变换、 Hadamard 变换等正交变换,有 $T = N, \Phi^T = \Phi$ 。若 $T > N$ ,
则称 $a$ 是 $x$ 的过完备表示。 过完备表示看似浪费,但由于字典中的
原子更多,可以更好地匹配信号 $x$ 中的各种特征,因此往往能得到更稀疏的近似
表示。在许多实际情况下,必须使用过完备字典,才能得到较好的稀疏表示。

显然,过完备表示不是唯一的。为了找到最稀疏的过完备表示,需要求解下列优化
问题:
\begin{problem}[$\ell_0$ 优化问题]
\begin{equation}
\hat a = \mathop{\arg\min}_a \|a\|_0 \quad s.t. \quad \Phi a = x
\end{equation}
\end{problem}
对于有噪声的情况,松弛等式约束,得到:
\begin{problem}[子集最优化问题] \label{prob:SubsetOpt}
\begin{equation}
\hat a = \mathop{\arg\min}_a  \|a\|_0 \quad s.t. \quad \|x - \Phi a\|_2 < \epsilon
\end{equation}
\end{problem}
然而,问题 \ref{prob:SubsetOpt} 是一个 NP 难度问题 \cite{GJ79} \cite{SubsetOptNPC},
这类问题的高效精确求解是困扰计算机科学界多年的难题,我们不应贸然挑战这一
难题。为此,只能近似求解问题 \ref{prob:SubsetOpt} 。一种近似方法是,用具有
凸性的范数 $\ell_1$ 或 $\ell_2$ 代替 $\ell_0$ ,得到:
\begin{problem}[Lasso 回归问题] \label{prob:Lasso}
\begin{equation} \label{eqn:Lasso}
\hat a = \mathop{\arg\min}_a \|a\|_1 \quad s.t. \quad \|x - \Phi a\|_2 < \epsilon
\end{equation}
\end{problem}
\begin{problem}[岭回归问题] \label{prob:Ridge}
\begin{equation}
\hat a = \mathop{\arg\min}_a \|a\|_2 \quad s.t. \quad \|x - \Phi a\|_2 < \epsilon
\end{equation}
\end{problem}
实际上,求解问题 \ref{prob:Ridge} 并不能得到较为稀疏的解。这是由于
$\ell_2$ 球是一个超球,具有旋转不变性,其解一般不位于坐标轴上。但是对于
问题 \ref{prob:Lasso} 来说, $\ell_1$ 球是一个正轴形,其角位于坐标轴上。
在合理的约束条件下,问题 \ref{prob:Lasso} 的解将出现在正轴形的角上,这就
保证了解的稀疏性。

以上是对稀疏性和稀疏表示的基本介绍。图 \ref{Fig:Compress} 描述了利用稀疏表
示进行信号压缩的简要过程。

\begin{figure}
\centering
\fbox{
\begin{tikzpicture}[node distance=6cm]
\node (n0) {原始图像};
\node [right of=n0] (n1) {可压缩表示};
\node [right of=n1] (n2) {近似稀疏表示};
\draw[->] (n0.east) -- node [above] {正交变换} node [below] {或求解问题 \ref{prob:Lasso}} (n1);
\draw[->] (n1.east) -- node [above] {取 $K$ 个最大元素} (n2);
\end{tikzpicture}
} % fbox

\caption{信号压缩简要流程}
\label{Fig:Compress}
\end{figure}

在使用过完备表示的图像压缩算法中,我们通过求解问题 \ref{prob:Lasso} ,近似
求解了问题 \ref{prob:SubsetOpt},从而从 $N$ 个采样数据计算出 $T > N$ 个组合
系数 (即过完备表示)。如果我们可以进一步减少采样数据的个数至 $M < N < T$,
同时保证可以求解问题 \ref{prob:SubsetOpt} 而得到原图像的近似稀疏表示,
就能够降低采样的时间和设备成本。显然, $M$ 不能无限减小,
否则将导致问题 \ref{prob:SubsetOpt} 出现病态。
\begin{definition}[良定问题] \label{def:well-posed}
若一个优化问题存在唯一解,且该解在输入数据连续变化时也连续变化,则称该
优化问题在 Hadamard 意义下良定。
\end{definition}
\begin{definition}[病态问题] \label{def:ill-posed}
若一个优化问题的解不满足定义 \ref{def:well-posed} 要求的存在性、唯一性或者
连续性要求,则称该问题是病态 的。 \cite{MathProblemImage}
\end{definition}

另外,即使问题在数学上是良定的,在实际上,它的解可能并不是正确的过完备表示。
例如,若简单地随机选取 $M$ 个点进行采样,即有 $(N-M)$ 个缺失的采样点。这是
数字图像处理中经典的图像修补问题,其良定性早已得到证明\cite{inpainting}。
 然而,采样缺失部分的图像细节显然会丢失。

显然,只有对于特定的矩阵 $\Phi$ ,才能使得问题 $\ref{prob:Lasso}$ 良定,并
给出一个稀疏的 $\hat{a}$ ,使得 $\hat{x} = \Phi \hat{a}$ 是原图像的良好
近似。Cand\`es 和 Tao 等人对 $\Phi$ 必须满足的性质进行了研究,在稀疏表示
理论的基础上建立了压缩感知理论。

\subsection{采样矩阵 $\Phi$ 的特性}

一般来说,研究 $\Phi x = y$ 解集的性质,将从 $\Phi$ 的零空间 入手:
\begin{definition}[零空间]
矩阵 $\Phi$ 的零空间 定义为:
\begin{equation}
\mathcal{N}(\Phi) = \{z:\Phi z = 0\}
\end{equation}
即所有左乘 $\Phi$ 后得到零向量的向量组成的集合。
\end{definition}

对于任意稀疏信号 $a$ ,如果希望能够基于测量值 $y = \Phi a$ 唯一地重建 $a$,
则对于任意不同向量 $a_1, a_2 \in \Sigma_K$ ,
必有 $\Phi a_1 \neq \Phi a_2$,即:
\begin{equation}
\Phi (a_1 - a_2) \neq 0
\end{equation}
由于 $a_1 - a_2$ 可能是 $\Sigma_{2K}$ 中的任意非零向量,仅当
\begin{equation}
\Sigma_{2K} \cap \mathcal{N}(\Phi) = \{0\}
\end{equation}
时,即零空间 $\mathcal{N}(\Phi)$ 不包含非零的 $2K$-稀疏向量,
才能唯一地重建 $a$。

下面考虑非严格稀疏(可压缩)的向量。此时零空间不仅不能包含 $2K$-稀疏向量,
也不应
包含可以压缩的近似稀疏向量。设 $\Lambda \subset \{1, 2, \cdots, N\}$ 是
一组索引, $\Lambda^c = \{1,2,\cdots,N\} \backslash \Lambda$ 是其补集。
对于向量 $x$ 来说,我们保留其中下标属于 $\Lambda$ 的元素,而把下标属于
$\Lambda^c$ 的元素置为 $0$ 。据此,可以检验零空间内是否存在近似稀疏的
向量:
\begin{definition}[$K$ 阶零空间特性]
若存在常数 $C > 0$ ,使得对于所有 $h \in \mathcal{N}(\Phi)$ 和所有
$|\Lambda| \leq K$ 的 $\Lambda$ 都有
\begin{equation} \label{eq:NSP}
\|h_\Lambda\|_2 \leq C \frac{\|h_{\Lambda^c}\|_1}{\sqrt{K}}
\end{equation}
则称矩阵 $\Phi$ 满足 $K$ 阶零空间特性 (NSP)。
\end{definition}
NSP 的意义是,零空间中的向量 $h$ 的值并不集中在少数下标上。例如,若 $h$
是严格 $K$-稀疏的,则存在 $\Lambda$ 使得 $\|h_{\Lambda^c}\|_1 = 0$ 。
此时式 \ref{eq:NSP} 要求 $\|h_\Lambda\|_2 = 0$。因此,若 $A$ 满足 $K$
阶 NSP,则零空间 $\mathcal{N}(A)$ 中唯一的 $K$-稀疏向量是 $0$。

% 下列定理表明, NSP 是以较小误差成功恢复 $x$ 的必要条件:
% \begin{theorem} \label{th:NSP}
% 若 $\Delta:\mathbb{R}^{M} \rightarrow \mathbb{R}^T$ 是一个从 $M$ 个测量
% 恢复 $N$ 个系数 $a$ 的算法。若结果的误差由下式限定:
% \begin{equation}
% \|\Delta(\Phi x) - x\|_2  \leq  \frac{C}{\sqrt{K}}
% \min_{\hat x \in \Sigma_K}\|\hat x - x\|_1
% \end{equation}
% 则 $\Phi$ 满足 $2K$ 阶 NSP 。
% \end{theorem}
% \begin{proof}
% 假设 $h \in \mathcal{N}(\Phi)$,且 $\Lambda$ 是 $h$ 中最大的 $2K$ 个元素
% 的下标。将 $\Lambda$ 划分为 $\Lambda_0, \Lambda_1$, 使得
% $|\Lambda_1| = |\Lambda_2| = K$。令 $x = h_{\Lambda_1} + h_{\Lambda^c}$,
% $x' = -h_{\Lambda_0}$。由于 $x'$ 是严格 $K$-稀疏的,有 $\Delta(\Phi x')
% = x'$。注意到 $x - x' = h \in \mathcal{N}(\Phi)$ ,有:
% \begin{equation}
% Ah = A(x - x') = 0
% \end{equation}
% 即 $\Phi x' = \Phi x$ ,因此 $x' = \Delta(\Phi x)$。可知:
% \begin{equation}
% \|h_\Lambda\|_2 \leq \|h\|_2 = \|x - x'\|_2 = \|x - \Delta(\Phi x)\|_2
% \leq \frac{C}{\sqrt{K}} \min_{\hat x \in \Sigma_K} \|\hat x - x\|_1
% \leq \frac{\sqrt{2}C}{\sqrt{2K}} \|h_{\Lambda^c}\|_1
% \end{equation}
% \end{proof}
% 

NSP 保证了定义
\ref{def:well-posed} 中解的唯一性要求。然而,实际信号存在噪声和量化误差,
因此定义 \ref{def:well-posed} 还要求解的连续性。即,当输入 $y$ 发生微小
变化时,结果 $\hat a$ 应当也发生微小变化,而不应突变。这就要求更为严格的
重建条件。 Cand\`es 和 Tao 引入了下面的约束等距特性,
用以描述在有噪情况下稀疏信号可恢复的条件:
\begin{definition}[约束等距特性] 若存在 $\delta_k \in (0,1)$ ,使得:
\begin{equation}
(1 - \delta_K) \|a\|_2^2 \leq \|\Phi a\|_2^2 \leq (1 + \delta_K) \|a\|_2^2
\end{equation}
对所有 $a \in \Sigma_K$ 都成立,则称矩阵 $\Phi$ 满足 $K$ 阶\emph{约束等距
特性} (RIP)。使得 $\Phi$ 满足 $K$ 阶 RIP 的最小 $\delta_K$ 称为
$\Phi$ 的约束等距常数。
\end{definition}

若矩阵 $\Phi$ 满足 $2K$ 阶 RIP ,则对于任意 $a_1, a_2 \in \Sigma_K$,有:
\begin{equation}
(1 - \delta_K) \|a_1 - a_2\|_2^2 \leq \|\Phi a_1 - \Phi a_2\|_2^2
\leq (1 + \delta_K) \|a_1 - a_2\|_2^2
\end{equation}
若 $\delta_K$ 很小,则表明,任意两个 $K$-稀疏的向量经过 $\Phi$ 的线性变换
后,它们的欧几里得距离几乎不变。因此,若输入 $x = \Phi a$ 只受到微小扰动,
对应的 $a$ 也应只受到微小扰动。

实际上,RIP 比 NSP 更加严格。
\begin{theorem} \label{th:RIPimpliesESP}
若矩阵 $\Phi$ 满足 $\delta_K \leq \sqrt{2} - 1$ 时的 K 阶 RIP,则
$\Phi$ 必然满足 $2K$ 阶 NSP,且式 \ref{eq:NSP} 中的常数
\begin{equation}
C = \frac{\sqrt{2} \delta_{2K}}{1 - (1 + \sqrt{2}) \delta_{2K}}
\end{equation}
\end{theorem}

定理 \ref{th:RIPimpliesESP} 的详细证明非常复杂,超出了本文的范围,参见参考
文献 \cite{IntroCS}。

在 RIP 的基础上,Cand\`es 指出,问题 \ref{prob:Lasso} 的解是一个误差较小
的近似稀疏表示。
\begin{theorem} \label{th:l1recovery}
若测量矩阵 $\Phi$ 满足 $2K$ 阶 RIP,且约束等距系数 $\delta_{2K} < 
\sqrt{2} - 1$,且观测结果
\begin{equation} \label{eqn:observe}
y = \Phi a + z
\end{equation}
中的噪声 $z$ 满足 $\|z\|_2 \leq \epsilon$,则 Lasso 回归的结果
\begin{equation}
\hat a = \mathop{\arg\min}_{a'} \|a'\|_1 \quad s.t. \quad \|y-\Phi a'\|_2
\leq \epsilon
\end{equation}
满足
\begin{equation}
\|\hat a - a\|_2 \leq C_0 K^{-1/2} \|a - a_{\Lambda}\|_1 + C_1 \epsilon
\end{equation}
其中 $\Lambda$ 是 $a$ 中最大的 $K$ 个元素的下标集合,$C_0$ 和 $C_1$ 是
与 $\delta_{2K}$ 有关的较小的常数:
\begin{equation}
C_0 = 2 \times \frac{1 - (1-\sqrt{2}) \delta_{2K}}{1 - (1+\sqrt{2})
\delta_{2K}}, \quad C_1 = 4 \times \frac{\sqrt{1 + \delta_{2K}}}
{1 - (1 + \sqrt{2}) \delta_{2K}}
\end{equation}
\end{theorem}

定理 \ref{th:l1recovery} 的证明非常复杂,参见参考文献 \cite{RIPimplieCS}。
该定理从理论上表明了通过求解 Lasso 回归 (问题 \ref{prob:Lasso}),可以近似
地重构稀疏表示,且误差是有限的。

如果能够构造矩阵 $\Phi$ ,使其以约束等距常数 $\delta_{2K} < \sqrt{2}-1$
满足 $2K$ 阶 RIP ,即可用该矩阵作为压缩感知的测量矩阵。实践上,一般采用
随机方法构造矩阵 $\Phi$ 。具体地,将 $\Phi$ 分解为两个矩阵的乘积
\begin{equation}
y = \Phi a = H \Psi a = H x
\end{equation}
其中 $\Psi$ 是稀疏字典,以保证 $x$ 的稀疏性;而 $H$ 是一个 $M \times T$
的矩阵,其中 $T$ 是原子的个数, $M$ 是采样点数。在实际测量中,信号
$x = \Psi a$ 是物理上真实存在的信号,而 $H$ 是作用在物理信号 $x$ 上的测量,
称 $H$
为物理测量矩阵。 $\Phi = H \Psi$ 是数学上作用于稀疏信号 $x$ 的测量,
为了方便,下面统一将 $\Phi$ (而不是 $H$)称为测量矩阵。已经证明,若 $x$ 足够
稀疏(稀疏度 $K$ 和 $M$ 相比足够小),字典 $\Psi$ 充分不相关,且 $H$ 是一个
随机矩阵,则 $\Phi = H \Psi$ 以高概率满足 $2K$ 阶 RIP 性质。因此,压缩感知
中的物理测量 $H$ 往往是随机矩阵。
\begin{definition}[相关性]
字典 $\Psi$ 的相关性 定义为:
\begin{equation}
\mu(\Psi) = \max_{i \neq j} \frac{|\psi_i^T \psi_j|}{\|\psi_i\|_2
\|\psi_j\|_2}
\end{equation}
即字典中任意两个不同原子(列向量)归一化内积绝对值的最大值。
\end{definition}

文献 \cite{CSRedundant} 证明了下列定理:
\begin{theorem} \label{th:RandomRIP}
若 $x \in \Sigma_K$ ,$2K \leq \mu(\Psi)^{-1}/16+1$ ,$H$ 是元素独立
同分布的随机矩阵,且采样点数 $M$ 足够大:
\begin{equation}
M \geq C_1(2K \log \frac{N}{2K} + 5.57 + t)
\end{equation}
其中 $C$ 是和 $H$ 的元素分布有关的常数,则 $\Phi = H \Psi$ 以概率
$1 - e^{-t}$ 满足 $\delta_{2K} \leq 1/3$ 的 $2K$ 阶 RIP。
\end{theorem}

由于 $1/3 < \sqrt{2} - 1$ ,该定理表明,矩阵 $\Phi$ 满足定理
\ref{th:l1recovery} 的要求,可以作为压缩感知的测量矩阵。另外, $t$ 在指数项
上,表明成功恢复的概率在理论门限附近迅速从 $0$ 增加到接近 $1$。

图 \ref{fig:CS} 描述了压缩感知的采样和稀疏恢复过程。

\begin{figure}
\centering
\fbox{
\begin{tikzpicture}[node distance=6cm]
\node (n0) {$x = \Psi a$};
\node [right of=n0] (n1) {$H\Psi a = \Phi a$};
\node [right of=n1] (n2) {$\hat a = \Delta(\Phi a)$};
\draw[->] (n0.east) -- node [above] {混叠欠采样} node [below] {物理测
量矩阵 $H$} (n1);
\draw[->] (n1.east) -- node [above] {稀疏恢复} node [below] {恢复算法
$\Delta$} (n2);
\end{tikzpicture}
} % fbox
\caption{压缩感知简要流程}
\label{fig:CS}
\end{figure}

我们已经讨论了测量矩阵 $\Phi$ 的性质。下面简要介绍常用的压缩感知稀疏
恢复算法。

\subsection{压缩感知稀疏恢复算法简介}

在压缩感知的框架下,测量值 $y$ 由式 \ref{eqn:observe} 给出,其中 $\Phi$
为测量矩阵, $z$ 为噪声。稀疏恢复算法的任务就是,根据给定的 $y$ 和 $\Phi$,
恢复出原始的稀疏信号 $x$。目前主流的稀疏恢复算法可分为凸优化类算法、贪心
算法、组合重建算法和贝叶斯算法这四类。下面简要介绍其中最为流行的凸优化类算
法和贪心算法。

\paragraph{凸优化类算法} 假设 $J(x)$ 是凸函数,且目标信号稀疏时, $J(x)$
的值较小。对于有噪声的测量值,为了重建稀疏信号 $x$ ,可以通过最小化 $J(x)$
促进 $x$ 的稀疏性,同时通过距离函数 $H$ 施加约束,使得 $\Phi x$ 接近测量值
$y$:
\begin{equation} \label{eqn:convexopt}
\hat x = \mathop{\arg \min}_{x'} J(x) \quad s.t. \quad H(\Phi x, y) \leq
\epsilon
\end{equation}
若选定 $J(x) = \|x\|_1$,$H(\Phi x, y) = \|\Phi x - y\|_2$,
则式 \ref{eqn:convexopt} 就成为式 \ref{eqn:Lasso},即 Lasso 回归问题
(问题 \ref{prob:Lasso})。该问题在数学和统计学上具有悠久的历史, Lasso
回归也是压缩感知领域非常重要的方法。在压缩感知领域, Lasso 回归又被称为
基追踪降噪 (BPDN) 问题。

对于图像来说,图像本身并不具有稀疏性,必须使用一稀疏字典 $\Psi$ 发掘其稀疏
性,才能通过 Lasso 或称 BPDN 求解。然而,在数字图像处理领域,往往可以假设
图像的梯度(即边缘)是稀疏的,并据此由测量值 $y$ 恢复 $x$。
\begin{definition} [全变分]
信号 $x$ 的全变分定义为:
\begin{equation}
TV(x) = \| |D(x)| \|_1
\end{equation}
即信号梯度的模的 $\ell_1$ 范数。
\end{definition}

根据之前对 $\ell_1$ 范数的讨论,最小化 $TV(x)$ 可以得到梯度稀疏的 $x$ 。
在式 \ref{eqn:convexopt} 中取 $J(x) = TV(x)$ ,即得到下列问题:
\begin{problem} \label{prob:TVopt}
\begin{equation}
\hat x = \mathop{\arg \min}_{x'} TV(x') \quad s.t. \quad 
\|\Phi x - y\|_2 \leq \epsilon
\end{equation}
\end{problem}
全变分最小化方法在核磁共振成像(MRI)领域的压缩感知问题中表现良好,这是由于
MRI 图像本身的梯度非常稀疏。然而,如果试图使用全变分最小化方法恢复梯度并不
稀疏的 $x$ ,将会产生强烈的伪迹。

凸优化是一个古老的研究课题,然而,已有的凸优化工具包往往并不适合压缩感知
问题。这是由于压缩感知中往往要面对高维信号,对优化求解的效率要求很高,
常规的优化算法无法满足压缩感知稀疏恢复的要求。例如,一般求解
\ref{prob:Lasso} 的方法是将其规约为二阶锥规划(SOCP) 问题,该问题
是现代优化方法研究的主要目标之一。但是,在压缩
感知稀疏恢复中,由于变量数和方程个数很大,SOCP 求解器运行十分缓慢。因此,
研究压缩感知的学者往往需要重新思考和开发新的优化算法。

\paragraph{贪心算法} 不同于求解巨大的全局优化问题,贪心算法在每一步迭代中
都选取目前看似最优的方案,期望通过局部最优解而产生全局最优解。

对于压缩感知中的稀疏信号恢复问题,最基本的贪心算法是匹配跟踪算法
(MP)。 在 MP 算法开始时,$x$ 的初步估计值 $x' = 0$,残差
$r = y - \Phi x'$从字典 $\Phi$ 中选取一个与 $y$ 相关性最大的列
$\phi_\lambda$ 。若相关系数为 $c$,就将组合系数 $x'_\lambda$ 增加 $c$,
同时残差将减小 $c \phi_\lambda$。反复执行这一步骤,直到残差的范数小于某个
预先设定的阈值,该算法终止。

在压缩感知的研究过程中, MP 算法被改进为正交匹配跟踪 (OMP)、逐步正交匹配
跟踪 (StOMP)、压缩感知匹配跟踪 (CoSaMP)、正则化正交匹配跟踪 (ROMP)、
等更加优秀的贪心算法。贪心算法的运行速度比全局凸优化快得多,然而抗噪声性能
很差,极少量的噪声就会导致恢复失败。因此,在实际应用中,往往只能使用
凸优化方法。
