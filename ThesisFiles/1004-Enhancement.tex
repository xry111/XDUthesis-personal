\chapter{光学成像辅助的压缩感知缺陷检测}

\section{问题描述}

文献 \cite{XDUCLBIC} 指出,在基于压缩感知的缺陷检测系统中,太阳能电池表面
栅状电极覆盖的区域的光电转化效率为 $0$ 。这些线状区域的存在严重破坏了信号
的稀疏性。对于全变分最小化方法,求梯度完全不能降低线状目标的稀疏性;
对于 $\ell_1$ 最小化方法,必须寻找一个能够稀疏表示直线的字典,才能降低
栅状电极的稀疏性。目前确实有这样的字典存在,即脊波(Ridgelet)
和曲波(Curvelet)变换\cite{ridgelet, curvelet} 。 十分巧合的
是,这两种变换正是由压缩感知理论的提出者 Cand\`es 和 Donoho 发明的。
然而,脊波和曲波变换的数学形式较为复杂,程序运行较慢,对于迭代次数较多的
优化问题而言,会极大增加优化算法的运行时间。

本文采用另一种方案,解决栅状电极带来的稀疏性退化问题。由于栅状电极在光学
上是可见的,我们可以用光学相机获取栅状电极的位置。从信息论的角度出发,当
我们获得了栅状电极的位置信息后,如果我们能够合理地运用这些信息,就可能
减少压缩采样过程需要获取的信息量,从而减少采样次数,提高检测效率。本章
的后续内容将表明,这的确是可行的。

\section{数学模型}

考虑问题的物理本质,栅状电极覆盖区域的光电转化效率为 $0$ ,纯粹是由于它
被电极覆盖的缘故,与电极之下的硅材料毫无关系。假如栅状电极是透明的(现实中
并没有这样良好的电极材料),它对测量结果没有影响,得到的测量结果是稀疏的。
然而,由于栅状电极的遮挡,被遮挡区域的测量值直接变为 $0$ ,导致稀疏
性退化。这样,自然想到采用图像修补算法,去除图像上的遮挡,尽可能
恢复未被遮挡的,稀疏的图像,从而提高压缩感知的效率和准确性。

下面给出图像修补问题的数学描述:
\begin{problem}[图像修补]
设有一矩阵 $H = diag\{W\}$,$W \in \{0,1\}^{N}$。设 $x$ 是一维向量
化的图像,则被遮挡并有噪声 $z$ 污染的图像为
\begin{equation}
y = Hx + z
\end{equation}
已知 $y$ 和 $H$ ,恢复未被遮挡的图像 $x$。
\end{problem}

和压缩感知恢复问题一样,图像修补问题亦可通过求解全变分最小化或 $\ell_1$
最小化解决,下面以全变分最小化为例说明。
\begin{equation}
\hat x = \mathop{\arg\min}_{x} TV(x) \quad s.t. \quad
\|Hx - y\|_2 \leq \epsilon
\end{equation}
注意虽然图像修补问题和压缩感知恢复问题都能够用全变分最小化方法求解,但其
内涵截然不同。在压缩感知理论中,若矩阵 $H$ 满足 RIP ,则可以认为测量
结果包含足够信息,因此能够以有限误差恢复 $x$ 。然而,在图像修补问题中,矩阵
$H$ 连 NSP 都不满足,更不会满足 RIP 。从信息论的角度来说,图像中被遮挡部分
的信息已经完全丢失,只能通过其他位置的信息,结合实际图像梯度稀疏的性质,采
用全变分最小化推测被遮挡部分的值。图 \ref{fig:BarbaraMask} 是在著名的
Barbara 图像上模拟遮挡栅状电极后的恢复结果,可以看出其中的直线状模糊痕迹,
即为被遮挡的部分。

\begin{figure}
\centering
\begin{subfigure}[t]{1.5in}
\includegraphics[width=1.5in]{Figure/barbara.png}
\caption{原始图像}
\end{subfigure}
\begin{subfigure}[t]{1.5in}
\includegraphics[width=1.5in]{Figure/barbara_mask.png}
\caption{被遮挡的图像}
\end{subfigure}
\begin{subfigure}[t]{1.5in}
\includegraphics[width=1.5in]{Figure/barbara_out.png}
\caption{恢复的图像}
\end{subfigure}
\caption{Barbara 图像的遮挡和恢复结果}
\label{fig:BarbaraMask}
\end{figure}

尽管如此,我们仍可以借鉴图像修补问题的思想,用于压缩感知问题。我们称图像修补
问题的遮挡矩阵为 $H_1$ ,压缩采样的测量矩阵为 $H_2$。对于图像修补,求解全
变分最小化问题可以恢复原图像,直观地看,这是由于 $H_1$ 的零空间包含被遮挡的
点,优化算法可以自由改变它们的值,以达到减小全变分的目的。同样的,在我们的
压缩感知理论中,鉴于无法确定被遮挡的值,同样应该将这些点加入零空间,使其
可以自由变化。为了达到这一目的,采用矩阵
\begin{equation}
H = H_2 H_1
\end{equation}
作为新的测量矩阵。其中,$H_1$ 的零空间包含所有被遮挡的像素点,赋予了优化算
法足够的自由度,可以通过调整这些点的值尽量降低全变分,使它们不对最终的目标
函数产生较大影响。 $H_2$ 则满足 RIP ,因此其不涉及被遮挡点的子矩阵同样满足
RIP,使得未被遮挡的像素点能够稳定、准确地被恢复。

\section{系统仿真}

测试用例如图 \ref{fig:opttestdata} 所示。图 \ref{fig:opttestdata:in}
是带有栅状电极遮挡的,具有清晰边缘(稀疏梯度)的坏块图像,图
\ref{fig:opttestdata:finger} 是单独提取出的栅电极图层,用于模拟光学成像
确认的栅电极覆盖区域。

\begin{figure}
\centering
\begin{subfigure}[h]{1.5in}
\includegraphics{Figure/testdata/2dsharp_finger.png}
\caption{被遮挡的坏块}
\label{fig:opttestdata:in}
\end{subfigure}
\begin{subfigure}[h]{1.5in}
\includegraphics{Figure/testdata/finger.png}
\caption{栅状电极遮挡区域}
\label{fig:opttestdata:finger}
\end{subfigure}
\caption{光学辅助压缩感知缺陷检测测试用例}
\label{fig:opttestdata}
\end{figure}

采用全变分最小化恢复得到的结果如图 \ref{fig:opttv} 所示。可见,将遮挡区域
加入零空间后,只需要 $10\%$ 至 $20\%$ 的采样,即可较好地恢复出坏块。比较
未改进的结果(图 \ref{fig:TVfinger},需要 $50\%$ 采样),可见这一改进对算法
的恢复能力有较大的提升作用。

\begin{figure}
\centering
\begin{subfigure}[h]{1.1in}
\includegraphics{Figure/TV/opt1.png}
\caption{$M = 0.1N$}
\end{subfigure}
\begin{subfigure}[h]{1.1in}
\includegraphics{Figure/TV/opt1f.png}
\caption{叠加栅电极}
\end{subfigure}
\begin{subfigure}[h]{1.1in}
\includegraphics{Figure/TV/opt2.png}
\caption{$M = 0.2N$}
\end{subfigure}
\begin{subfigure}[h]{1.1in}
\includegraphics{Figure/TV/opt2f.png}
\caption{叠加栅电极}
\end{subfigure}
\caption{采用光学成像辅助的全变分最小化恢复仿真结果}
\label{fig:opttv}
\end{figure}

此外,$\ell_1$ 范数最小化恢复也可以用完全相同的方法加以改进,只要用
$H_2 H_1$ 作为测量矩阵,即可消除栅电极的
影响。和改进前需要 $50\%$ 测量相比,改进后只需要 $20\%$ 测量即可取得
较好的恢复结果,如图 \ref{fig:optl1} 所示。

\begin{figure}
\centering
\begin{subfigure}[h]{1.1in}
\includegraphics{Figure/L1/opt2.png}
\caption{$M = 0.2N$}
\end{subfigure}
\begin{subfigure}[h]{1.1in}
\includegraphics{Figure/L1/opt2f.png}
\caption{叠加栅电极}
\end{subfigure}
\begin{subfigure}[h]{1.1in}
\includegraphics{Figure/L1/opt3.png}
\caption{$M = 0.3N$}
\end{subfigure}
\begin{subfigure}[h]{1.1in}
\includegraphics{Figure/L1/opt3f.png}
\caption{叠加栅电极}
\end{subfigure}
\caption{采用光学成像辅助的全变分最小化恢复仿真结果}
\label{fig:optl1}
\end{figure}

\section{关于系统工程实现的讨论}

我们已经完成了系统的数学模型和数学仿真,下面从工程实现的角度讨论系统的
结构设计。在光学成像辅助的压缩感知缺陷检测系统中,我们使用 CCD 相机采集
太阳能电池板表面的光学图像,然后经二值化处理就能得到栅状电极覆盖的区域,
即矩阵 $H_1$ 的对角元。

在工程实践中,我们需要确定 CCD 相机相对于结构光的位置基准。为此,我们用
投影仪代替 LCD
屏幕作为结构光源。在正式开始测量前,投影仪首先在太阳能电池表面投出一幅
结构光。 CCD 相机对太阳能电池表面反射的结构光进行采集,采集结果与计算机
中存储的结构光进行相关匹配。由于结构光是 Bernoulli 矩阵的一行,即随机的
$0-1$ 信号,因此取相关最大的位置,就是位置基准。

根据测试点大小和栅状电极宽度的相对关系,我们讨论三种情况。首先,对于超大幅
面检测,测试点大小远大于栅状电极的宽度,此时即可忽略栅状电极对测试点的遮挡,
恢复结果上完全看不到栅状电极。其次,对于小面积样品的精确检测,测试点大小
远小于栅状电极的宽度,可以认为所有测试点不是没有被栅状电极覆盖,就是完全
被栅状电极覆盖,不存在半覆盖的情况。最后,对于栅状电极宽度和测试点大小接近
的情况,我们可以简单地认为那些被栅状电极部分覆盖的测试点也被完全覆盖。即,
我们放弃栅状电极附近的缺陷信息。在最坏情况下,每条栅电极等价于遮挡了三倍
于其宽度的区域。

在测量过程中,同恢复过程一样,我们使用 $H_2 H_1$ 作为测量矩阵。 
$H_1$ 的存在保证了结构光不会照射被遮挡区域,也不会对周围的正常区域产生
漏光、衍射等影响。

\section{本章小结}

本章讨论了采用光学相机辅助,确定栅状电极遮挡区域,并利用遮挡区域的信息修正
测量矩阵,从而消除栅状电极对稀疏恢复过程的不利影响的方法。

我们提出的这一方法的主要优点是,数学上,
遮挡 $H_1$ 和待测的 $x$ 是相乘关系,这和实际情况是一致的。如果它们是加减
关系,结果就未必正确。例如,有一很大的缺陷,占据了太阳能电池的大片面积,
如果试图通过加法补偿栅状电极的遮挡,则结果如图 \ref{fig:addmask} 所示。
可见,缺陷内部被遮挡的区域受到过度补偿,产生大量白线,同样破坏了稀疏性。

\begin{figure}[h]
\centering
\begin{subfigure}[h]{1.1in}
\includegraphics{Figure/add1.png}
\caption{被遮挡的图像}
\end{subfigure}
\begin{subfigure}[h]{1.1in}
\includegraphics{Figure/add2.png}
\caption{补偿栅状电极的图像}
\end{subfigure}
\begin{subfigure}[h]{1.1in}
\includegraphics{Figure/add3.png}
\caption{实际补偿结果}
\end{subfigure}
\caption{加法补偿遮挡的不合理性。}
\label{fig:addmask}
\end{figure}

我们的方法基本保持了基于压缩感知的缺陷检测方法不包含机械部件,快速、准确
的优点,且测量效率比原始的压缩感知方法更高。但我们引入了 CCD 相机和投影仪,
它们包含光学镜头,需要对焦的过程,还要注意可能存在的光学失真问题。为此,
在科研实验中,可以使用投影仪和商用相机,而在实际制造测量仪器时,应优化整合
光源和 CCD 相机,从而减少对焦时间,同时降低系统的重量和成本。
