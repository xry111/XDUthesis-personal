\chapter{绪论}

目前,随着人们环保意识的提高,太阳能电池被越来越多地运用于生产和生活的各个
领域。然而,由于太阳能电池内部半导体材料的特性,以及较大的面积、较为复杂的
生产过程,太阳能电池内部不可避免地存在一些缺陷。缺陷严重影响太阳能电池的
光电转化效率。快速、准确地检测太阳能电池中的缺陷有利于及时对生产线上的次品
进行过滤、返修,并可以根据缺陷的形态推断其产生原因,进而改进工艺,从而达到
提高产品质量、降低产品成本的目的。

\section{问题背景}

1954 年 4 月 25 日, Bell 实验室成功制造了第一块实用的太阳能电池
\cite{First}。 1958 年,美国陆军通信兵实验室在 T. Mandelkorn 的率领下制造出
抗辐射能力更强的 “ n 在 p 上” 型太阳能电池,随后美国海军研究实验室将其应
用于先锋 1 号人造卫星,开创了在航天器使用太阳能电池的先例\cite{Vanguard}。
20 世纪 70 年代末期,受国际政治、经济环境影响,西方国家难以进口足够的石油,
油价大幅上涨。为寻求替代能源,西方发达国家开始了大量使用太阳能电池的探索。

早在 1960 年代,人们就已经注意到了硅材料晶体缺陷引起的太阳能电池缺陷问题,
并用 X 射线衍射技术加以研究\cite{SiliconXray}。 1980 年,著名的加州理工学院
喷气推进实验室 (JPL) 为美国国家航天局 (NASA) 设计了一台太阳能电池缺陷
检测仪器。该仪器采用激光束对太阳能电池进行扫描,以此确定缺陷的位置
\cite{JPLDefectAnalyzer}。

传统的太阳能电池缺陷检测系统需要使用光束对太阳能电池的感光面进行完整扫描,
其测量时间正比于感光面的面积。随着工艺进步,太阳能电池的幅面也不断提高,
传统缺陷检测系统的检测时间大大增加。更为雪上加霜的是,对光束进行偏转,基本
依赖于对光源或光学元件的机械转动,导致检测系统的效率难以进一步提高。

为了提高太阳能电池的检测效率,人们又提出了一系列利用光学成像完成的间接检测
方法,其基本思想是对太阳能电池施加外部激励(如光、电、热等),之后采用
光学成像系统检测太阳能电池板的微弱荧光。这种方法的测量时间很短,一般能够
在数秒内完成测量,但它们依据的原理是太阳能电池板缺陷区域内载流子浓度的变化。
而实际上,载流子浓度异常未必导致缺陷,缺陷也未必由载流子浓度异常引发,因此
成像方法存在漏报、误报缺陷的问题,往往需要应用各种后处理算法甚至人工判读
图像,才能锁定缺陷。

近年来,在医学成像、雷达信号处理、天文观测等领域,都出现了采样成本过高,
传统的 Nyquist 采样无法实现或者效率低下的问题。在处理这类问题的过程中,以
E. Cand\`es、 T. Tao 和 D. Donoho 为代表的一些科学家综合稀疏表示理论、优化
理论、矩阵论、概率论、泛函分析等,提出了一种从少量采样数据恢复具有稀疏表示
的信号的理论框架,这就是压缩感知。

注意到,对太阳能电池的逐像素扫描实际上就是一个采样过程,这一过程花费的
巨大时间正是一个采样成本过高,导致 Nyquist 采样十分缓慢的问题。如果能够
在太阳能电池的缺陷检测中应用压缩感知理论,就可能极大地提高太阳能电池
缺陷检测的效率,替代或辅助基于光学成像的间接方法,提高缺陷检测的准确率,
使得缺陷能尽早发现,尽早修复或返工,从而提高太阳能电池的良品率,进一步
促进光伏产业的发展。

\section{研究现状}

\subsection{现有的太阳能电池缺陷检测方法}

目前,主要的太阳能电池缺陷检测方法可分为采用粒子束或波束完整扫描太阳能
电池感光面的扫描法,以及采用外部激励和光学成像进行间接检测的
成像法。

扫描法包括光感应电流 (LBIC) 法和电子束感应电流 (EBIC)法。
LBIC 法即使用窄光束对太阳能电池的感光面进行扫描,从而得到太阳能电池上各个
小面积元的光电转化效率,转化效率很低的区域即为缺陷\cite{LBIC}。 LBIC 法的
原理和测量过程非常直观,而且可以通过改变光束的波长、强度,以及测量过程中
在太阳能电池上施加的偏压,获取多组数据进行比较,从而更加全面地认识太阳能
电池的性能指标\cite{LBICEnc}。

EBIC 法则使用扫描电子显微镜 (SEM) 的电子束对太阳能电池进行扫描,并测量
感应电流的强度\cite{EBIC}。
这种方法的空间分辨率非常高,是在实验室测试太阳能电池的最佳方法。
但是,受到 SEM 镜头大小的限制,这种方法的扫描区域很小。此外,
SEM 的电子束在空气中会迅速衰减,必须在真空环境下工作。这些缺点限制了 EBIC
法在工业界的应用。

两种扫描法的共同缺点是,波束(粒子束)必须对整个太阳能电池的感光面进行扫描,
才能找到所有的缺陷。特别是 LBIC 法中,光束难以用非机械手段进行偏转,机械部
件的缓慢移动限制了提高测量效率的努力。为了避免扫描,可以使用成像法。

成像法则是向太阳能电池施加外部激励,使太阳能电池发出电磁辐射,再
采用感光阵列对辐射进行成像,最后对获得的图像进行数学处理,以得到缺陷的分布
情况。常用的非扫描法包括光致荧光法 (PL),电致荧光法 (EL) ,以及锁相热
图法(LIT)。

在 PL 法中,首先在电池片的感光面施加一激发光源。经照射后,太阳能电池内部产
生电子-空穴对。电子-空穴复合将产生红外荧光,由于缺陷部位的能级分布情况和电
导率等电学参数与正常部位不同,成为载流子的强复合中心,导致载流子密度较小,
荧光强度较小。在暗室中采用 CCD 相机即可采集荧光图像,图像中较暗的区域即为
缺陷部位。 PL 法是一种快速非接触测量方法,能够在生产线的各个工艺阶段检测
太阳能电池片(包括未安装电极的电池片),适用于生产线的实时监测
\cite{FastPL}。 此外,如果采用多光谱成像技术,收集红外荧光的频谱信息,就能够
利用缺陷部位辐射波长的变化,更好地定位、识别缺陷\cite{SpectrePL}。 但是,
PL 法只能检测硅片中的缺陷,对于减反射膜损坏、电极断裂等缺陷毫无检测能力。

EL 法是近年来比较流行的太阳能电池检测方法。在 EL 法中,太阳能电池被施加一定
的正向偏置电流,由于耗尽区很薄,少数载流子可以越过耗尽区,进入扩散区,并与
扩散区的多数载流子复合发光。和 PL 法的情形一样,缺陷部位的少数载流子密度较
小,在 EL 图像中产生暗斑,从而被检测出来\cite{EL}。 和 PL 法相比,EL 法除了
能够检测硅片中的缺陷以外,还能检测电极故障导致的缺陷(如断栅、虚焊等),
目前市售太阳能电池缺陷检测设备大多使用 EL 法。

PL 法和 EL 法都是通过检测载流子复合时产生的红外荧光对缺陷进行诊断,因此只能
检验与载流子的产生和复合有关的缺陷(如硅片微裂纹、预击穿点),对于其他缺陷
(例如减反射膜缺损或沉积过厚)则无能为力。另外,其他一些不属于缺陷的情况也
能产生荧光,例如硅片中含有的杂质原子、多晶硅中的晶界都能产生荧光条纹。因此,
往往需要人工判读荧光图像,才能找出其中的缺陷。人工判读增加了工作量,而且引
入了较强的主观性,对缺陷检测是不利的。

LIT 法则是在太阳能电池板上施加脉冲电压,对太阳能电池进行热成像,温度分布反
映出太阳能电池内部的电流密度,由此可以找到硅片内部的分路电流缺陷和电极的缺
陷。但是, LIT 不能检测其他类型的缺陷,而且不能进行实时快速检测,丧失了成像
方法的主要优越性。

在以上的方法中,只有 LBIC 方法能够直接给出电池板各个区域的光电转化效率,而
其他方法都是间接方法,可能漏报或者误报缺陷。然而, 正如前文所述, LBIC 的测
量速度受限于机械扫描速度,非常缓慢。另外,由于测量时间很长,在测量过程中,
环境温度等测量条件会不可避免地发生变化,引入了额外的变量,在测量得到的图像
上产生带状伪迹。为解决这一问题,最近,人们提出将压缩感知 (CS) 理论应用
于 LBIC 测量。下面对压缩感知理论及其在太阳能电池缺陷检测中的应用进行介绍。 

\subsection{压缩感知理论的提出和发展}

压缩感知理论是近年来提出的新型采样理论。2004 年, E. Cand\`es 为解决
核磁共振成像(NMI)采样时间过长,需要病人长时间保持特定姿势不动的问题,
试图采用 $\ell_1$ 范数最小化方法,从频域欠采样得到的模糊图像重建一幅
更为清晰的图像。然而,他惊奇地发现,重建图像与完整采样得到的原始图像
完全一致,没有任何差别。随后, Cand\`es 与 Tao 和 Donoho 等人展开合作,
最终证明了对于具有稀疏表示或近似稀疏表示的信号来说,在一定条件下,有可能
通过少量线性混叠的采样,较为精确地恢复原始信号\cite{CS2006}。

此后,压缩感知迅速在多个领域得到应用。 Rice 大学团队根据压缩感知理论,设计
了著名的单像素相机,直观地展示了压缩采样在现实中的可行性,且有望用于红外、
太赫兹等特殊波段的成像\cite{SinglePixel}。 Eldar 团队将压缩感知用于 A/D
转换,实现了远低于 Nyquist 频率对信号的信息采样\cite{CSAD}。 
欧空局为解决星载计算机计算
能力不足和星地通信缓慢的问题,在 HERSCHEL 星载望远镜使用压缩感知理论,卫星
直接将欠采样数据传回地面,由地面进行稀疏信号恢复,从而得到成像结果
\cite{HERSCHEL}。

\subsection{压缩感知在太阳能电池缺陷检测中的应用现状}

最近几年,为解决 LBIC 法测量效率低下的问题,人们将压缩感知理论用于太阳能
电池缺陷检测。

英国国家物理实验室 (APL) 团队构造了基于压缩感知的太阳能电池
缺陷检测系统原型,以数字光处理 (DMD) 芯片实现对太阳能电池各位置光电效率的
的混叠测量,之后以 $\ell_1$ 最小化方法恢复出各个位置的光电转化效率
\cite{CLBIC16, CLBIC17} 。

西安电子科技大学团队构造了另一款原型,与 APL 原型不同,该原型以液晶屏幕
作为结构光源实现混叠测量,以全变分最小化方法恢复光电转化效率
\cite{XDUCLBIC}。同 $\ell_1$ 最小化相比,对于边界清晰的缺陷,
全变分最小化在采样数据个数相对不足时,能够更好地恢复出缺陷的大致轮廓和
位置。

目前来说,压缩感知在太阳能电池缺陷检测的应用尚处于起步阶段。上述两款原型已
经取得了一定成果,但它们仅仅迈出了第一步,仍然存在各自的不足,需要大量改进
才能真正用于工业生产和科学研究。本文将
主要针对太阳能电池栅状电极破坏测量结果稀疏性的问题提出改进方案,从而进一步
提高测量效率,凸显压缩感知方法相比于传统方法的优越性。

\section{论文内容及结构}

本文讨论了太阳能电池的工作原理、内部结构、缺陷的形态,以及常用的缺陷检测
方法;详细介绍了压缩感知理论;讨论了使用压缩感知方法进行缺陷检测的理论基础,
介绍了现有的压缩感知缺陷检测原型装置;给出了在压缩感知测量框架下进一步
降低采样次数和测量时间的方案。

论文的内容安排如下:

第一章为绪论,简要介绍了太阳能电池缺陷检测的重要意义和现有研究结果,给出了
本文的内容简介和章节安排。

第二章简要描述太阳能电池的工作原理、内部结构,分析了常见的太阳能电池缺陷的
产生原因和形态;详细介绍压缩感知的理论框架,简要介绍压缩感知中常用的重建
算法。

第三章进行压缩感知缺陷检测方法的理论分析,论证了压缩感知方法的可行性和相对
于传统 LBIC 方法的优势,对压缩感知缺陷检测过程进行了数值仿真。

第四章介绍现有的压缩感知缺陷检测原型装置,评估了其工作表现。

第五章针对现有压缩感知缺陷检测方法对太阳能电池板栅状电极处理不利的问题,对
现有方案进行改进,并进行数值仿真,说明改进方法的优越性。

第六章总结了本论文的工作成果,并提出进一步改进的方向。
