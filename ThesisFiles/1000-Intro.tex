\chapter{绪论}

目前,随着人们环保意识的提高,太阳能电池被越来越多地运用于生产和生活的各个
领域。然而,由于太阳能电池内部半导体材料的特性,以及较大的面积、较为复杂的
生产过程,太阳能电池内部不可避免地存在一些缺陷。缺陷严重影响太阳能电池的
光电转化效率。快速、准确地检测太阳能电池中的缺陷有利于及时对生产线上的次品
进行过滤、返修,并可以根据缺陷的形态推断其产生原因,进而改进工艺,从而达到
提高产品质量、降低产品成本的目的。

\section{问题背景与研究现状}

1954 年 4 月 25 日, Bell 实验室成功制造了第一块实用的太阳能电池。
\cite{First} 1958 年,美国陆军通信兵实验室在 T. Mandelkorn 的率领下制造出
抗辐射能力更强的 “ n 在 p 上” 型太阳能电池,随后美国海军研究实验室将其应
用于先锋 1 号人造卫星,开创了在航天器使用太阳能电池的先例。\cite{Vanguard}
20 世纪 70 年代末期,受国际政治、经济环境影响,西方国家难以进口足够的石油,
油价大幅上涨。为寻求替代能源,西方发达国家开始了大量使用太阳能电池的探索。

早在 1960 年代,人们就已经注意到了硅材料晶体缺陷引起的太阳能电池缺陷问题,
并用 X 射线衍射技术加以研究。\cite{SiliconXray} 1980 年,著名的加州理工学院
喷气推进实验室 (JPL) 为美国国家航天局 (NASA) 设计了一台太阳能电池缺陷
检测仪器。该仪器采用激光束对太阳能电池进行扫描,以此确定缺陷的位置。
\cite{JPLDefectAnalyzer}

此后,人们又提出了多种太阳能电池缺陷检测方法。目前流行的方法有:

\paragraph{扫描法} 包括光感应电流 (LBIC) 法和电子束感应电流 (EBIC)法。
LBIC 法即使用窄光束对太阳能电池的感光面进行扫描,从而得到太阳能电池上各个
小面积元的光电转化效率,转化效率很低的区域即为缺陷。\cite{LBIC} LBIC 法的
原理和测量过程非常直观,而且可以通过改变光束的波长、强度,以及测量过程中
在太阳能电池上施加的偏压,获取多组数据进行比较,从而更加全面地认识太阳能
电池的性能指标。\cite{LBICEnc}

EBIC 法则使用扫描电子显微镜 (SEM) 的电子束对太阳能电池进行扫描,并测量
感应电流的强度。\cite{EBIC}
这种方法的空间分辨率非常高,是在实验室测试太阳能电池的最佳方法。
但是,受到 SEM 镜头大小的限制,这种方法的扫描区域很小。此外,
SEM 的电子束在空气中会迅速衰减,必须在真空环境下工作。这些缺点限制了 EBIC
法在工业界的应用。

两种扫描法的共同缺点是,波束(粒子束)必须对整个太阳能电池的感光面进行扫描,
才能找到所有的缺陷。特别是 LBIC 法中,光束难以用非机械手段进行偏转,机械部
件的缓慢移动限制了提高测量效率的努力。为了避免扫描,可以使用成像法。

\paragraph{成像法} 向太阳能电池施加外部激励,使太阳能电池发出电磁辐射,再
采用感光阵列对辐射进行成像,最后对获得的图像进行数学处理,以得到缺陷的分布
情况。常用的非扫描法包括光致荧光法 (PL),电致荧光法 (EL) ,以及锁相热
图法(LIT)。

在 PL 法中,首先在电池片的感光面施加一激发光源。经照射后,太阳能电池内部产
生电子-空穴对。电子-空穴复合将产生红外荧光,由于缺陷部位的能级分布情况和电
导率等电学参数与正常部位不同,成为载流子的强复合中心,导致载流子密度较小,
荧光强度较小。在暗室中采用 CCD 相机即可采集荧光图像,图像中较暗的区域即为
缺陷部位。 PL 法是一种快速非接触测量方法,能够在生产线的各个工艺阶段检测
太阳能电池片(包括未安装电极的电池片),适用于生产线的实时监测。
\cite{FastPL} 此外,如果采用多光谱成像技术,收集红外荧光的频谱信息,就能够
利用缺陷部位辐射波长的变化,更好地定位、识别缺陷。\cite{SpectrePL} 但是,
PL 法只能检测硅片中的缺陷,对于减反射膜损坏、电极断裂等缺陷毫无检测能力。

EL 法是近年来比较流行的太阳能电池检测方法。在 EL 法中,太阳能电池被施加一定
的正向偏置电流,由于耗尽区很薄,少数载流子可以越过耗尽区,进入扩散区,并与
扩散区的多数载流子复合发光。和 PL 法的情形一样,缺陷部位的少数载流子密度较
小,在 EL 图像中产生暗斑,从而被检测出来。\cite{EL} 和 PL 法相比,EL 法除了
能够检测硅片中的缺陷以外,还能检测电极故障导致的缺陷(如断栅、虚焊等),
目前市售太阳能电池缺陷检测设备大多使用 EL 法。

PL 法和 EL 法都是通过检测载流子复合时产生的红外荧光对缺陷进行诊断,因此只能
检验与载流子的产生和复合有关的缺陷(如硅片微裂纹、预击穿点),对于其他缺陷
(例如减反射膜缺损或沉积过厚)则无能为力。另外,其他一些不属于缺陷的情况也
能产生荧光,例如硅片中含有的杂质原子、多晶硅中的晶界都能产生荧光条纹。因此,
往往需要人工判读荧光图像,才能找出其中的缺陷。人工判读增加了工作量,而且引
入了较强的主观性,对缺陷检测是不利的。

LIT 法则是在太阳能电池板上施加脉冲电压,对太阳能电池进行热成像,温度分布反
映出太阳能电池内部的电流密度,由此可以找到硅片内部的分路电流缺陷和电极的缺
陷。但是, LIT 不能检测其他类型的缺陷,而且不能进行实时快速检测,丧失了成像
方法的主要优越性。

在以上的方法中,只有 LBIC 方法能够直接给出电池板各个区域的光电转化效率,而
其他方法都是间接方法,可能漏报或者误报缺陷。然而, 正如前文所述, LBIC 的测
量速度受限于机械扫描速度,非常缓慢。另外,由于测量时间很长,在测量过程中,
环境温度等测量条件会不可避免地发生变化,引入了额外的变量,在测量得到的图像
上产生带状伪迹。为解决这一问题,最近,人们提出将压缩感知 (CS) 理论应用
于 LBIC 测量。\cite{CLBIC16} \cite{CLBIC17}

LBIC 测量的结果是一幅图像 $I(x,y)$ ,该图像反映太阳能电池板缺陷的分布情况。
尽管这不是传统意义上的光学图像,但它仍然具有图像的一般性质,实践上也往往用
图像的相关理论和方法对太阳能电池板的缺陷分布图像进行处理。 CS 理论是近年来
流行的一种图像采集理论,采用 CS 方法,可以大大减小图像采集的采样点数目。对
于常规的光学图像采集,减小采样点数目只有在某些特定情况下有价值。然而,对于
LBIC 方法,测量时间和采样点数目成正比关系,采用 CS 方法能够较好地解决 LBIC
测量速度缓慢的问题。

\section{论文内容及结构}

本文讨论了太阳能电池的工作原理、内部结构、缺陷的形态,以及常用的缺陷检测
方法;详细介绍了压缩感知理论;讨论了使用压缩感知方法进行缺陷检测的理论基础,
介绍了现有的压缩感知缺陷检测原型装置;给出了在压缩感知测量框架下进一步
降低采样次数和测量时间的方案。

论文的内容安排如下:

第一章为绪论,简要介绍了太阳能电池缺陷检测的重要意义和现有研究结果,给出了
本文的内容简介和章节安排。

第二章简要描述太阳能电池的工作原理、内部结构,分析了常见的太阳能电池缺陷的
产生原因和形态;详细介绍压缩感知的理论框架,简要介绍压缩感知中常用的重建
算法。

第三章进行压缩感知缺陷检测方法的理论分析,论证了压缩感知方法的可行性和相对
于传统 LBIC 方法的优势,对压缩感知缺陷检测过程进行了数值仿真。

第四章介绍现有的压缩感知缺陷检测原型装置,评估了其工作表现。

第五章针对现有压缩感知缺陷检测方法对太阳能电池板栅状电极处理不利的问题,对
现有方案进行改进,并进行数值仿真,说明改进方法的优越性。

第六章总结了本论文的工作成果,并提出进一步改进的方向。
