% 中文摘要
\begin{abstract}
本文介绍了太阳能电池的基本理论和工艺结构,描述了太阳能电池中的缺陷,总结了
传统的太阳能电池缺陷检测方法的优缺点;详细介绍了压缩感知理论;
介绍了近年来基于压缩
感知理论检测太阳能电池缺陷的新型方法,完成了基于压缩感知的太阳能电池缺陷
检测的数值仿真;介绍了现有的基于压缩感知的太阳能电池缺陷检测原型装置,及
其实现过程中遇到的困难。

针对太阳能电池栅状电极影响测量结果可压缩性,造成压缩感知检测方法的效率难以
进一步提高的问题,本文分别考虑栅状电极和实际的缺陷,将栅状电极视为对缺陷
分布图像的遮挡。通过光学成像获得栅状电极的遮挡区域,借鉴对于图像遮挡的修补
算法,对压缩感知的稀疏恢复算法进行改进,在求解过程中利用已知的遮挡区域信息,
同时修补图像遮挡,从而消除恢复结果中的栅状电极,增强结果的稀疏性,从而达到
降低采样次数,提高检测效率的目的。
\end{abstract}
\keywords{太阳能电池, 缺陷, 压缩感知, 稀疏性, 图像修补}

% 英文摘要
\begin{enabstract}
At first the basic theory, the processing and the structure of solar cells
are introduced.  The compressive sensing theory, and the new method to
detect solar cell defects with it are introduced in detail.  The numerical
emulation of this method is completed.  The existing prototypes detecting
solar cell defects with compressive sensing are introduced, and the
difficulties implementing them are discussed.

The fingers and busbars on the solar cells are affecting the sparsity of
solar cell defect image notably.  The time effiency of compressive
sensing method is limited by them.  To solve this issue, the fingers and
busbars and normal defects are considered seperately.  The fingers and
busbars are regarded as occlude areas of the defect image, and can be
acquiststioned with a optical camera.  Referencing the image inpainting,
the sparse signal recovery algorithms in compressive sensing are altered
to input the known occlude areas and inpaint them simutaniously.  The
result image without fingers and busbars are more sparse so the sampling
number could be reduced and the time effiency can be inproved.
\end{enabstract}
\enkeywords{solar cells, defects, compressive sensing, sparsity, image inpainting}
